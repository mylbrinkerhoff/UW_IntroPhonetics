\documentclass[12pt]{article}
\usepackage[utf8]{inputenc}
\usepackage{enumitem}
\usepackage{geometry}
\geometry{margin=1in}
\usepackage{titlesec}
\titleformat{\section}{\normalfont\Large\bfseries}{}{0em}{}
\titleformat{\subsection}{\normalfont\large\bfseries}{}{0em}{}
\begin{document}
Mystery Language Project Part A Guidelines (Undergraduate)\\
You will complete an individual project on a language that you do not know. The goal of this project is to introduce you to resources and methods used by phoneticians and to help you hone your transcription skills. Do not worry about making mistakes in your initial transcriptions. This project is about the journey and the skills you pick up and not about getting the immediate correct answers.  Deadlines for the corresponding steps outlined below are in the course calendar.  We will complete the preparation step on the first day of class.\\
\section*{Part A:  Transcription and Charts Due 2/19/2024}
Sound files.  You will be given sound files for an unlabeled language that you do not know.\\
Each sound file is one word said by a native speaker, and the file name is the English gloss (translation) for the meaning of that word.\\
Most collections are organized into folders labeled consonants, vowels, and tone when applicable.  Some may have separate folders for diphthongs or other relevant distinctions, but the absence of a grouping does not necessarily imply the absence of a distinction (e.g., you may find diphthongs in your vowels folder).  In your consonants folder, you will have enough evidence to identify all the consonant distinctions in your language (all phonemes and possibly some allophones, although you may not have enough evidence to determine whether a sound is an allophone).  In your vowels folder, you will have at least one instance of each vowel phoneme.\\
The number of files does not necessarily correspond directly to the number of phonemes in your language; i.e., not every sound file has a unique sound, but it is possible that you will only have one example of a given sound.\\
Transcriptions.  Transcribe your sound files in a list/table format.  You may want to begin doing this by hand, but the draft you turn in must be typed using correct IPA symbols.\\
\begin{itemize}
\item It is highly recommended that you use high-quality headphones, preferably circum-aural (enclosing your ears), to listen to your files in a quiet environment while you transcribe.
\end{itemize}
Computer speakers often distort sounds, and many earbuds have low quality output.  The Language Learning Center in the basement of Denny Hall has good headphones, and other computer labs on campus may also.\\
Your transcriptions should appear in neat tables with column headings and use Unicode fonts (e.g., Charis SIL or Doulos SIL) with standard IPA symbols and conventions.  (If you use a word processor, make a table or use uniform tab stops throughout to create the appearance of columns).\\
Column headings should include, in order: word number, word gloss, your transcription.  (You may add others, as appropriate; e.g., if you’d like to add second options or notes to unconfident transcriptions.)\\
You should have sections with headings for each folder (consonants, vowels, tones, etc.).\\
\subsection*{You have two options for how to order your sound files within each table section/folder}
Present your transcriptions in the order the words appear in their folders – if the files are numbered, order your transcriptions numerically; if they are not, order them alphabetically.\\
Or, you may rearrange the words in each section to more clearly show phonetic distinctions.  For example, you may want to put minimal pairs together or put words that illustrate each consonant sound in the order they appear in your chart.  If your files are numbered, include the number in the first column of your table, even though the numbers will be out of order.  (This makes it easier to find each sound file later.)\\
Phonetic inventory charts.  From your transcriptions, build phonetic inventory charts for your language.  Every symbol used in your transcriptions should appear in your charts and vice versa.  Your final drafts should be clear (not blurry) and typed using Unicode IPA fonts and conventions, arranged in neat tables with row and column headings for articulatory descriptions.  You should not have any empty rows/columns, and the font style and size should be uniform and of a similar size to your prose.  Add footnotes to distinguish pairs, e.g. voiced/voiceless and rounded/unrounded.\\
Consonants.  Follow the main IPA consonant chart, placing in the appropriate locations only those sounds you observe in your sound files.  You may add rows or columns as appropriate to fully describe your inventory, or you may add more than two symbols to a cell.  Examples: Add rows for affricates, implosives, palatalized consonants; add columns for labialized consonants after corresponding place columns; or just add the appropriate symbols to existing cells.\\
Vowels.  Follow the IPA vowels chart, placing the vowels you transcribe in locations appropriate to their articulatory descriptions. Your chart doesn’t have to be slanted for the front vowels; it can be a simple rectangular table.  Note: You only need to include tense/lax in descriptions if there are pairs that are distinguished only by this, or if you can see a pattern where tense vowels appear in different environments than lax.\\
Non-quality distinctions.  If your language has phonemic distinctions other than quality (length, nasalization, creakiness, breathiness, laryngealization, etc.), decide whether to create separate charts for each set.  If all vowel qualities participate in the distinction, you may simply state this in words below the chart.  Otherwise, it often works to add appropriate symbols next to the plain vowels in the same chart.  For example, inventories with length distinctions often place long vowels next to their short counterparts.  If only a subset of vowels is affected, it may be clearer to show them in a separate chart.\\
Diphthongs (and triphthongs, if applicable). If your language has a small vowel inventory, you may put diphthongs on the same chart as monophthongs. You can put the diphthong at the location of any vowel involved, or you can draw an arrow from the first vowel to the second vowel (and to the third vowel, for triphthongs), and write the diphthong symbol at either vowel location. If this would be crowded, create a separate chart. If you do not have a separate diphthongs folder, attempt to identify any diphthongs on your own.\\
Tone, pitch accent, etc.  If your language has lexical tone, or if you have evidence of grammatical tone or pitch accent, include a chart with example words to illustrate the distinction.  (Ask me (your instructor) for examples if necessary.)\\
Below each set of charts (consonants, vowels, tones, etc.), write brief prose descriptions of all the sounds identified (1-2 paragraphs per chart).  You must use complete articulatory descriptions and list all symbols used in your charts, but sounds should be described in groups to reduce repetition.  Examples: “The language has voiceless and voiced stops at three places of articulation: bilabial [p, b], alveolar [t, d], and velar [k, ɡ]”; or “The language has bilabial, alveolar, and velar stops, both voiceless [p, t, k] and voiced [b, d, ɡ].”\\
\begin{itemize}
\item While finishing your first attempt at your transcriptions, charts, and descriptions, you may want to meet with the other student(s) working on languages of the same language family (in person or virtually through chat, email, or discussion board).  Ask each other for help identifying sounds and transcribing difficult words, and look over each other’s charts and descriptions for internal consistency.  However, you do not need to agree on particular sounds or symbol choices, and your descriptions should not be worded identically.
\end{itemize}
\subsection*{Turn in your transcriptions, charts, and descriptions (steps 3-5 above) in a PDF format. Note that your writing style should be appropriate for an academic paper}
Not conversational\\
It need not be extremely formal (e.g., you may use first person, active voice, contractions, etc.)\\
You should also use complete sentences, correct spelling, clear organization, varied sentence structure, and avoid slangs.\\
\section*{Part B: Comparisons and Research; Final Report Due 3/14/2024}
You will receive an article with the phonetic description of your language that corresponds to your sound files.  Read the article carefully and take notes or highlight information you’ll use in your report (see below).\\
Compare your charts, transcriptions, and descriptions to those in the article.  Where there are discrepancies, try to determine how significant they are.  Did you miss a distinction entirely, or were your descriptions close?  Listen to the words again to see if you can hear the distinctions in the article’s transcriptions.\\
Published papers could be “wrong”! If you disagree with the illustration paper, please describe your disagreement and explain why your transcription and/or analysis would be more reasonable. You can use additional sources to support your analyses.\\
\subsection*{Do additional research on your language.  Some articles include some of this information, but most only give an overview.  Use Ethnologue.com and other sources recommended by your instructor to find the following}
Your language’s three-letter ISO 639-3 code.\\
Its genetic classification (language family, genus, and a step or two below or a lower classification that is well known).\\
Genetic cousins (languages most closely related and/or well-known), plus comments about dialects.  If your language is an isolate, try to find comments about any hypotheses about its origins, relatives, or previous classifications.\\
The geographic area(s) where it is spoken, plus social details about its speakers, if applicable (e.g., a particular ethnicity or social class).\\
The populations of native speakers and second-language users, plus the spheres in which it is used, if applicable/possible (e.g., used in trade, government, at home only, etc.).  Note if it is endangered and any comments about efforts to revive it.\\
If you have trouble finding any of this information, please get in touch with me so that I can give you guidance on where to look or what to say about a lack of existing data.\\
Keep track of your sources and compile a bibliography in JIPA or APA format (see the class lecture slides for examples and links, and run yours by the instructor if you’re unsure).\\
\begin{itemize}
\item Important:  Wikipedia is not an academic source.  It is not original research and should not be cited as such – it is more like a group book report.  However, it can be a useful tool for finding original sources.  If you find information on Wikipedia, you must verify it through the original source linked in the entry’s bibliography – and then cite that source, not Wikipedia.
\end{itemize}
\subsection*{Type and turn in your final report (with subject headings, 11-12 point font, 1-inch margins, double- or 1.5-spaced, as one PDF). Your page limit is 10 pages.  Include the following, roughly in this order (or with slight modifications, as appropriate or as approved ahead of time)}
An introduction to your language including the information you found in your outside research.  This must be in prose (1-2 paragraphs, not a list or chart) and use in-line citations in JIPA or APA format.\\
Your original phonetic inventory charts, each followed by revised descriptions. These should be revised for completeness and accuracy in describing only your charts, and to address any feedback from your draft.\\
Follow each chart + description with the corresponding charts from your article.  You cannot copy and paste the table or graph from the sources, but you can recreate them on your own. If your charts are very similar to those in the article, you may highlight these differences on your charts or describe them without including the article charts.  For example, if you missed a symbol, you can add it to your chart using a different color or by circling it, and then explain the difference in prose below the charts. If you chose a different label for a description (e.g., place of articulation column heading, laryngealization vs. pharyngealization), you may indicate the heading on the chart (with a circle, arrow, different font color, etc.) and then explain the difference in prose.\\
Summary and discussion of differences between your transcriptions and charts and those of the article.  Some bulleted lists or charts may be appropriate to illustrate differences, but you must also discuss them briefly in prose.  You should cover all large divisions (consonant place/manner distinctions, vowel features, etc.), even if your transcriptions were fairly accurate.\\
Note categorical differences and similarities as well as smaller details (e.g., “I was able to distinguish the front round vowels, but I did not identify both central vowels, and I only noticed nasality in a few words… I described the set of consonants [t, d, l] as alveolar, but the authors describe them as dental...  I missed the voiceless glottal fricative [h] at the ends of words, so I thought that [pah] and [pa] must have had a difference in vowel quality.”).  Also describe differences in conventions or notational systems (e.g., if you used tone numbers and they used lines, provide a table or description of how these systems correspond).\\
Especially discuss how the article treated the sounds that were difficult for you.  How close were you?  Does it use one of the options you considered?\\
\begin{itemize}
\item Note that you may choose to represent sections b), c), and d) in this document by the folders that divided your sound files; e.g., you may write a section for consonants (and then do what is asked in b), c), and d) in this document), then a section for vowels, then tones, etc.
\end{itemize}
Conclusion.  Briefly (1-2 paragraphs) summarize your paper and add a general comment evaluating the process (what you liked/didn’t, what was hardest, most enjoyable, etc.).   If you worked with a classmate at all, name them and briefly describe how you worked together and whether it was helpful.  (Did you help each other identify difficult sounds?  Did you check each other’s writing?  If you disagreed on a transcription or symbol, how did you resolve this?  Was one of you closer to correct? etc.)  You may also add constructive feedback that could be used to maintain or improve the project process for future classes (including things you think should not be changed).\\
Your original transcriptions with an additional column for corrections.  If your transcription of a word is identical to that in the article, do not enter a correction for that word.  If all of your transcriptions agree but with a consistent difference of a convention (e.g., you always used schwa whenever the article used barred-i), just describe the difference in prose below your transcriptions.  Note that you can use the article’s conventions for your corrections (e.g., if you used tone numbers but they used tone lines, you do not need to attempt to translate their line system to your numbers), but you should describe how the systems correspond in your discussion of differences.\\
Bibliography in JIPA or APA format.  See the links and examples on the class webpage.\\
You will lose points if you use another citation format (e.g., Chicago, Turabian, Vancouver, MLA)\\
Tips and things to look out for\\
Transcriptions\\
Many students find it helpful to handwrite their first transcription drafts.  It is faster than typing using correct IPA symbols and easier to keep track of changes and possibilities.  When you go to type it up, you may find errors, patterns, or inconsistencies that you were unable to see before.\\
Keep in mind that many tonal distinctions also co-occur with changes in voice quality or laryngeal setting, e.g., low dipping and super-low tones often also involve creaky voicing.\\
Breathiness and nasalization can often sound similar.  To distinguish the two, open your sounds in Praat to look for signs of nasalization (e.g. dampened formants, thick voice bar, etc.).\\
Charts and prose\\
Make sure to use the IPA [ɡ] symbol (not [g]), small-cap [ɪ] (not capital [I]), and that no symbols in your charts are automatically capitalized. Descriptions\\
Descriptors should appear in the same order as corresponding symbols, e.g.: “Language X has voiceless and voiced stops at three places of articulation: bilabial [p, b], alveolar [t, d], and velar [k, ɡ].” (In each pair, the voiceless symbol appears first, so “voiceless” should appear before “voiced.”)\\
Comparisons\\
Note that [a] is more often considered central rather than front.  So, if your article shows [a] in a central location on the vowel chart, but you used a different central symbol, you need only note the difference in symbol choices.  If you used [a] but described it as front, discuss how you thought that sound was front, while they characterize it as central.  General/Report\\
In your intro, refer to your language code like this: ISO 639-3 code: xxx. Replace “xxx” with your language’s three-letter code, in all lowercase or all caps.\\
You can type your whole document in a Unicode font (Charis SIL or Doulos SIL preferred).  Using one of these ensures your teacher will be able to read your symbols in a Word doc.  You can also convert your work to a PDF.  In Word 2007 and beyond, you can “Save as” a PDF.  If you use something else, you might try a free online converter such as http://en.pdf24.org/doc2pdf. • 	Spelling: liquids and glides are types of “approximant” not “approximate”\\
Use uniform line spacing throughout your document (double or 1.5).\\
Don’t forget in-line citations!  See the tips doc on the class webpage for formatting and some ready-made citations for commonly-used sources.\\
\end{document}