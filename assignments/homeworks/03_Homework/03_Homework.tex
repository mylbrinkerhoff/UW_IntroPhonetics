% !TEX TS-program = lualatex
% !TEX encoding = UTF-8 Unicode

\documentclass[12pt, letterpaper]{article}

%%BIBLIOGRAPHY- This uses biber/biblatex to generate bibliographies according to the
%%Unified Style Sheet for Linguistics
\usepackage[main=american, german]{babel}% Recommended
\usepackage{csquotes}% Recommended
\usepackage[backend=biber,
        style=unified,
        maxcitenames=3,
        maxbibnames=99,
        natbib,
        url=false]{biblatex}
\addbibresource{homework.bib}
\setcounter{biburlnumpenalty}{100}  % allow URL breaks at numbers
% \setcounter{biburlucpenalty}{100}   % allow URL breaks at uppercase letters
% \setcounter{biburllcpenalty}{100}   % allow URL breaks at lowercase letters

%%TYPOLOGY
\usepackage[svgnames]{xcolor} % Specify colors by their 'svgnames', for a full list of all colors available see here: http://www.latextemplates.com/svgnames-colors
\usepackage[hmargin=1in,vmargin=1in]{geometry}  %Margins          
\usepackage{graphicx}	%Inserting graphics, pictures, images 		
\usepackage{stackengine} %Package to allow text above or below other text, Also helpful for HG weights 
\usepackage{fontspec} %Selection of fonts must be ran in XeLaTeX or LuaLateX
\usepackage{amssymb} %Math symbols
\usepackage{amsmath} % Mathematical enhancements for LaTeX
\usepackage{setspace} %Linespacing
\usepackage{multicol} %Multicolumn text
\usepackage{enumitem} %Allows for continuous numbering of lists over examples, etc.
\usepackage{multirow} %Useful for combining cells in tablesbrew 
\usepackage{hanging}
\usepackage{fancyhdr} %Allows for the 
\pagestyle{fancy}
\fancyhead[L]{\textit{LING 450/550: Homework 3}}
\fancyhead[R]{Autumn 2025} 
\fancyfoot[L]{[Updated: \textit{\today}]}
\fancyfoot[C]{}
\fancyfoot[R]{\thepage} 
\renewcommand{\headrulewidth}{0.4pt}
\setlength{\headheight}{14.5pt} % ...at least 14.49998pt
% \usepackage{fourier} % This allows for the use of certain wingdings like bombs, frowns, etc.
% \usepackage{fourier-orns} %More useful symbols like bombs and jolly-roger, mostly for OT
\usepackage[colorlinks,allcolors={black},urlcolor={blue}]{hyperref} %allows for hyperlinks and pdf bookmarks
% \usepackage{url} %allows for urls
% \def\UrlBreaks{\do\/\do-} %allows for urls to be broken up
\usepackage[normalem]{ulem} %strike out text. Handy for syntax
\usepackage{tcolorbox}
\usepackage{datetime2}
\usepackage{longtable}
\usepackage{booktabs}

%%FONTS
\setmainfont{Libertinus Serif}
\setsansfont{Libertinus Sans}
\setmonofont[Scale=MatchLowercase]{Libertinus Mono}

%%PACKAGES FOR LINGUISTICS
\usepackage[noipa]{OTtablx} % Use this one generating tableaux without using TIPA
\usepackage{langsci-gb4e} % Language Science Press' modification of gb4e
\usepackage{tikz} % Drawing Hasse diagrams
\usepackage{leipzig} %	Offers support for Leipzig Glossing Rules

%%LEIPZIG GLOSSING FOR ZAPOTEC
\newleipzig{el}{el}{elder} % Elder pronouns
\newleipzig{hu}{hu}{human} % Human pronouns
\newleipzig{an}{an}{animate} % Animate pronouns
\newleipzig{in}{in}{inanimate} % Inanimate pronouns
\newleipzig{pot}{pot}{potential} % Potential Aspect
\newleipzig{cont}{cont}{continuative} % Continuative Aspect
\newleipzig{stat}{stat}{stative} % Stative Aspect
\newleipzig{and}{and}{andative} % Andative Aspect
\newleipzig{ven}{ven}{venative} % Venative Aspect
% \newleipzig{res}{res}{restitutive} % Restitutive Aspect
\newleipzig{rep}{rep}{repetitive} % Repetitive Aspect

%%TITLE INFORMATION
% \title{TITLE}
% \author{Mykel Loren Brinkerhoff}
% \date{\today}

%%MACROS
\newcommand{\sub}[1]{\textsubscript{#1}}
\newcommand{\supr}[1]{\textsuperscript{#1}}

\makeatletter
\renewcommand{\paragraph}{%
  \@startsection{paragraph}{4}%
  {\z@}{0ex \@plus 1ex \@minus .2ex}{-1em}%
  {\normalfont\normalsize\bfseries}%
}
\makeatother
\parindent=10pt


\begin{document}

%%If using linguex, need the following commands to get correct LSA style spacing
%% these have to be after  \begin{document}
    % \setlength{\Extopsep}{6pt}
    % \setlength{\Exlabelsep}{9pt}		%effect of 0.4in indent from left text edge
%%

%% Line spacing setting. Comment out the line spacing you do not need. Comment out all if you want single spacing.
%	\doublespacing
%	\onehalfspacing

\begin{center}
    {\Large \textbf{LING 450/550: Homework 3}}\\
    {\large Due: Tuesday, October 14, 2025 at 8:30am}
\end{center}
%\maketitle
%\maketitleinst
\thispagestyle{fancy}

% \tableofcontents

%------------------------------------
%\section{} \label{}
%------------------------------------

\textbf{Please read all instructions carefully before starting the assignment.} The instructions include important information for this homework. If you have any questions, please ask them in class or via email.

\begin{tcolorbox}[colback=LightGray!10!white,colframe=LightGray!75!black,title=Instructions]
    \begin{itemize}
        \item \textbf{Assignments must be written in a separate document. You cannot write on this PDF.} 
        \item Please submit your homework as a PDF file on Canvas by the deadline.
        \item You may work with other students in the class, but you must write up your solutions independently and in your own words. Please list any collaborators at the end of your assignment.
        \item You may use any resources you like, but please cite them appropriately using the \href{https://apastyle.apa.org/}{\textit{APA citation style}} or the \href{https://langsci-press.org/unifiedstylesheet}{Unified Style Sheet for Linguistics}. If you use online resources, please make sure they are reputable and reliable.
        \item Please make sure your solutions are clear and well-organized. Use headings, bullet points, and diagrams where appropriate to help illustrate your points.
        \item If you have any questions or need clarification on any part of the assignment, please don't hesitate to reach out to me or the TA via email or during office hours.
        \item \textbf{AI tools (e.g., ChatGPT, Grammarly, Copilot, etc.) can only be used to help with grammar, spelling, and formatting on this assignment. Any other use will be considered a violation of course policy and will result in a 0.}
    \end{itemize}
\end{tcolorbox}

%------------------------------------
\section*{Part 1: Question about acoustics and articulation} \label{sec:acoustics}
%------------------------------------

\begin{enumerate}
    \item If I were to artificially remove the harmonic corresponding to the fundamental frequency of a periodic signal, the fundamental frequency you hear will not perceptibly change to you. Write a few sentences explaining why.
    % \item Describe, in detail, the steps in producing the consonants in the middle of the phrase \textit{thi\underline{\textbf{ck sn}}ow}. You do not need to make illustrations, you only need to describe the sequence of gestures necessary in producing the sounds. 
    \item Describe and compare what happens to the wavelength of a 200Hz sine wave if it is traveling through Sulphur Dioxide (213m/s), Helium (965m/s) , and our vocal tracts (350m/s). This will require you to make use of the wavelength formula to answer this question.
    \item What is the frequency of a 1.5m $\lambda$ in Helium (Show your math)? How does this frequency compare to Sulphur Dioxide and to air in the vocal tract?
\end{enumerate}

%------------------------------------
\section*{Part 2: Spectra and spectrograms} \label{sec:spectra}
%------------------------------------

Study the spectrogram and waveform below, I took spectra from the slices (A-C) indicated. Which slice correspond to the spectra on the following page? Justify your answers by making specific reference to what you know about what spectra, waveforms, and spectrograms represent.

\begin{center}
    \includegraphics[width = \textwidth]{figs/SpectraSpectrogram.png}
\end{center}

\newpage
\begin{center}
    \includegraphics[width = 0.75\textwidth]{figs/Spectra_01.png}I
    \includegraphics[width = 0.75\textwidth]{figs/Spectra_02.png}II
    \includegraphics[width = 0.75\textwidth]{figs/Spectra_03.png}III
\end{center}




%------------------------------------
%BIBLIOGRAPHY
%------------------------------------

%\singlespacing
%\nocite{*}
\printbibliography[heading=bibintoc]

\end{document}