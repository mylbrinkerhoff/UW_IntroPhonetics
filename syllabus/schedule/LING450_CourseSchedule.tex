% !TEX TS-program = lualatex
% !TEX encoding = UTF-8 Unicode

\documentclass[12pt, letterpaper]{article}
\setlength{\headheight}{14.49998pt}

%%BIBLIOGRAPHY- This uses biber/biblatex to generate bibliographies according to the
%%Unified Style Sheet for Linguistics
\usepackage[main=american, german]{babel}% Recommended
\usepackage{csquotes}% Recommended
% \usepackage[backend=biber,
%         style=unified,
%         maxcitenames=3,
%         maxbibnames=99,
%         natbib,
%         url=false]{biblatex}
% \addbibresource{}
% \setcounter{biburlnumpenalty}{100}  % allow URL breaks at numbers
% \setcounter{biburlucpenalty}{100}   % allow URL breaks at uppercase letters
% \setcounter{biburllcpenalty}{100}   % allow URL breaks at lowercase letters

%%TYPOLOGY
\usepackage[svgnames]{xcolor} % Specify colors by their 'svgnames', for a full list of all colors available see here: http://www.latextemplates.com/svgnames-colors
\usepackage[hmargin=1in,vmargin=1in]{geometry}  %Margins          
\usepackage{graphicx}	%Inserting graphics, pictures, images 		
\usepackage{fontspec} %Selection of fonts must be ran in XeLaTeX or LuaLateX

\usepackage{amssymb} %Math symbols
\usepackage{amsmath} % Mathematical enhancements for LaTeX
\usepackage{setspace} %Linespacing
\usepackage{enumitem} %Allows for continuous numbering of lists over examples, etc.
\usepackage{multirow} %Useful for combining cells in tablesbrew 
\usepackage{fancyhdr} 
\usepackage{hyperref} %allows for hyperlinks and pdf bookmarks
\hypersetup{
colorlinks = true,
allcolors = Black,
    linkcolor = DarkBlue,
    citecolor = DarkGreen,
    urlcolor = DarkMagenta
}
\usepackage[normalem]{ulem} %strike out text. Handy for syntax
\usepackage[datesep=.]{datetime2}  % for date and time formatting
\usepackage{longtable}  % multipage tables
\usepackage{makecell}   % for manual line breaks
\usepackage{booktabs}   % for professional looking tables
\usepackage{titlesec}   % customize section titles
\usepackage{array}      % for better arrays (eg matrices) in maths
\usepackage{parskip}    % better handling of spacing between paragraphs


% FANCYHDR Settings
\pagestyle{fancy}
\fancyhf{}
\lhead{\textbf{\CourseCode: \CourseTitle}}
\rhead{\Semester}
\lfoot{\textit{Last updated: \today}}
\rfoot{\thepage}


% Customization placeholders
\newcommand{\CourseCode}{LING 450/550}
\newcommand{\CourseTitle}{Introduction to Linguistic Phonetics}
\newcommand{\Semester}{Autumn 2025}

\newcommand{\InstructorName}{Myke Brinkerhoff}
\newcommand{\InstructorEmail}{mlbrinke@uw.edu}
\newcommand{\InstructorOfficeHours}{W 3:30-4:30 (\href{https://washington.zoom.us/j/96351372694?pwd=QBe9YQv6JczhqbXIqw3bdsRbsKBfG1.1}{Zoom}); F 10:30-11:30 (GUG 407A); or by appointment }

\newcommand{\TAOneName}{Ty Gill-Saucier}
\newcommand{\TAOneEmail}{ztgill@uw.edu}
\newcommand{\TAOneOfficeHours}{M 3:30-4:30 (\href{https://washington.zoom.us/j/94177487053}{Zoom}); R 10:30-11:30 (GUG 407); or by appointment}

\newcommand{\TATwoName}{NAME}
\newcommand{\TATwoEmail}{XXX@uw.edu}
\newcommand{\TATwoOfficeHours}{TBD (LOCATION)}

\newcommand{\ClassTimeLocation}{TR 8:30--10:20am, CMU 120 and Zoom (550C students only)}

\newcommand{\InstructorInfo}{
    \InstructorName\ (\href{mailto:\InstructorEmail}{\InstructorEmail})\\
        Office: \InstructorOfficeHours
    }

\newcommand{\TAOneInfo}{
    \TAOneName\ (\href{mailto:\TAOneEmail}{\TAOneEmail})\\
    Office: \TAOneOfficeHours
    }

%%FONTS
\setmainfont{Libertinus Serif}
\setsansfont{Libertinus Sans}
\setmonofont[Scale=MatchLowercase]{Libertinus Mono}

%%MACROS
\newcommand{\sub}[1]{\textsubscript{#1}}
\newcommand{\supr}[1]{\textsuperscript{#1}}
\providecommand{\lsptoprule}{\midrule\toprule}
\providecommand{\lspbottomrule}{\bottomrule\midrule}
\newcommand{\fittable}[1]{\resizebox{\textwidth}{!}{#1}}

\renewcommand{\arraystretch}{1.3} % spacing between rows

\begin{document}


\begin{center}
    {\Large \textbf{Tentative Course Schedule (Subject to change)}} %\\
\end{center}
% \thispagestyle{fancy}

\begin{description}[style=multiline, leftmargin=6cm,font=\bfseries]
    \item[Ladefoged and Johnson (L\&J)] Ladefoged, Peter \& Keith Johnson. 2014. \textit{A course in phonetics}. 7th ed. Stamford, CT: Cengage Learning.
    \item[Reetz and Jongman (R\&J)] Reetz, Henning \& Allard Jongman. 2020. \textit{Phonetics: Transcription, production, acoustics, and perception} (Blackwell Textbooks in Linguistics). Second edition. Hoboken: John Wiley \& Sons.

    \item[Zsiga (Zs)] Zsiga, Elizabeth C. 2024. \textit{The sounds of language: An introduction to phonetics and phonology} (Linguistics in the World 3). 2nd ed. Hoboken, NJ: John Wiley \& Sons.
\end{description}

\begin{longtable}{|c|c|l|p{3cm}|p{1.25cm}|l|l|l|}

\hline
\multicolumn{2}{|l|}{\textbf{Date}} & \textbf{Readings} & \textbf{Homework} & \textbf{Quiz} & \textbf{IPA} & \textbf{Labs} & \textbf{Final Project} \\
\hline
\endfirsthead

\hline
\multicolumn{2}{|l|}{\textbf{Date}} & \textbf{Readings} & \textbf{Homework} & \textbf{Quiz} & \textbf{IPA} & \textbf{Labs} & \textbf{Final Project} \\
\hline
\endhead

\hline
\multicolumn{8}{r}{\textit{Continued on next page}} \\
\endfoot
\hline
\endlastfoot

% Reformatted course schedule table according to LaTeX style guide

\multicolumn{8}{|l|}{\textit{Week 0: Description vs. prescription; overview of phonetics}} \\ \hline
Thu & Sept 25 &  &  & & & & \\ \hline
Fri & Sept 26 &  &  & Quiz 0 & & & \\ \hline

\multicolumn{8}{|l|}{\textit{Week 1: Anatomy of speech; Overview of sound types}} \\ \hline
Tue & Sept 30 & R\&J Ch. 2 & Homework 1 & & & & \\ \hline
Thu & Oct 2   & & & & & & \\ \hline
Fri & Oct 3   & & & Quiz 1 & & & \\ \hline

\multicolumn{8}{|l|}{\textit{Week 2: Acoustic of sound}} \\ \hline
Tue & Oct 7   &  & Homework 2 & & & & \\ \hline
Thu & Oct 9   & R\&J Ch. 7 &  &  & & & \\ \hline
Fri & Oct 10  &  &  & Quiz 2 & & & \\ \hline

\multicolumn{8}{|l|}{\textit{Week 3: Acoustics of Vocoids}} \\ \hline
Tue & Oct 14  & L\&J Ch. 3 & Homework 3 & & & & \\ \hline
Thu & Oct 16  & R\&J Ch. 9 & & & Practice 1 & & \\ \hline
Fri & Oct 17  & & & Quiz 3 &  &  &  \\ \hline

\multicolumn{8}{|l|}{\textit{Week 4: Acoustics of Consonants}} \\ \hline
Tue & Oct 21  & L\&J Ch. 4 & Homework 4 & & & Lab 1 & \\ \hline
Thu & Oct 23  & R\&J Ch. 10 & & & Practice 2 & & \\ \hline
Fri & Oct 24  & & & Quiz 4 & & & \\ \hline

\multicolumn{8}{|l|}{\textit{Week 5: Suprasegmentals}} \\ \hline
Tue & Oct 28  &  & Homework 5 & & & Lab 2 & \\ \hline
Thu & Oct 30  & R\&J Ch. 11 & -- & & & & \\ \hline
Fri & Oct 31  & & & Quiz 5 & & & Part 1 \\ \hline

\multicolumn{8}{|l|}{\textit{Week 6: Speech Perception}} \\ \hline
Tue & Nov 4   &  & Homework 6 & & & Lab 3 & \\ \hline
Thu & Nov 6   & R\&J Ch. 13 &  & & Practice 3 & & \\ \hline
Fri & Nov 7   & & & Quiz 6 & & & \\ \hline

\multicolumn{8}{|l|}{\textit{Week 7: Auditory phonetics}} \\ \hline
Tue & Nov 11  & \multicolumn{6}{c|}{\textbf{No class (Veterans Day/Remembrance Day)}} \\ \hline
Thu & Nov 13  &  & Homework 7 & & & & \\ \hline
Fri & Nov 14  & & & Quiz 7 & & &  \\ \hline

\multicolumn{8}{|l|}{\textit{Week 8: Sociophonetics?}} \\ \hline
Tue & Nov 18  & & Homework 8 & & & & \\ \hline
Thu & Nov 20  & R\&J Ch. 13 & & & Practice 4 & & \\ \hline
Fri & Nov 21  & & & Quiz 8 & & & \\ \hline

\multicolumn{8}{|l|}{\textit{Week 9: Phonemic analysis? }} \\ \hline
Tue & Nov 25  & Zs Ch. 10 & Homework 9 & & & & \\ \hline
Thu & Nov 27  & \multicolumn{6}{c|}{\textbf{No class (Thanksgiving)}} \\ \hline

\multicolumn{8}{|l|}{\textit{Week 10: Variation and applications?}} \\ \hline
Tue & Dec 2   &  & & & & & \\ \hline
Thu & Dec 4   & & & &  & & \\ \hline
Fri & Dec 5   & & & Quiz 9 & & & Project Due \\ \hline

\multicolumn{8}{|c|}{\textbf{Final Exam Due: Tue Dec 9, 5:00 PM}} \\ \hline
\end{longtable}


%------------------------------------
%BIBLIOGRAPHY
%------------------------------------

%\singlespacing
%\nocite{*}
% \printbibliography[heading=bibintoc]

\end{document}