% !TEX TS-program = lualatex
% !TEX encoding = UTF-8 Unicode

\documentclass[professionalfonts]{beamer}
\usepackage{iftex,ifxetex}
\ifPDFTeX
  \usepackage[utf8]{inputenc}
  \usepackage[T1]{fontenc}
  \usepackage{lmodern}
\else
  \ifluatex
    \usepackage{unicode-math} 
    \defaultfontfeatures{Ligatures=TeX}
    % \setmathfont{Latin Modern Math}
    \setsansfont{CMU Sans Serif}
    % \setsansfont{Linux Biolinum O}
  \fi
\fi

\mode<presentation>
{
  \usetheme{Madrid} % or try Darmstadt, Madrid, Warsaw, ...
  % or ...
  \setbeamertemplate{bibliography item}{}
  \setbeamercovered{transparent}
  % or whatever (possibly just delete it)
}

\usepackage{fontspec}
\usepackage[english]{babel}
% or whatever
\usepackage{csquotes}
\usepackage[backend=biber,
        style=unified,
        maxcitenames=3,
        maxbibnames=99,
        natbib,
        url=false]{biblatex}
% \addbibresource{Dissertation.bib}
\setmainfont{Libertinus Sans} % Main font
% \setsansfont{libertinus sansserif} % Sans serif font
% \setmonofont{CMU Typewriter Text}
\renewcommand{\ttdefault}{cmtt}

% \usepackage[colorlinks,allcolors={black},urlcolor={blue}]{hyperref} %allows for hyperlinks and pdf bookmarks 
\usepackage{graphicx}	%Inserting graphics, pictures, images 		
\usepackage{multicol} %Multicolumn text
\usepackage{multirow} %Useful for combining cells in tablesbrew 
% \usepackage{booktabs} %Enhanced tables
% \usepackage{underscore} %Allows for underscores in text mode
% \usepackage[colorlinks,allcolors={black},urlcolor={blue}]{hyperref} %allows for hyperlinks and pdf bookmarks
\usepackage{url} %allows for urls
\def\UrlBreaks{\do\/\do-} %allows for urls to be broken up
% \usepackage[normalem]{ulem} %strike out text. Handy for syntax
% \usepackage{tcolorbox}
% \usepackage{datetime2}
\usepackage{caption}
\usepackage{subcaption}
\usepackage{langsci-gb4e} % Language Science Press' modification of gb4e
\usepackage{tikz} % Drawing Hasse diagrams
\usetikzlibrary{decorations.pathreplacing}
\usepackage{leipzig} %	Offers support for Leipzig Glossing Rules


\title[LING 450/550] % (optional, use only with long paper titles)
{Introduction to Linguistic Phonetics}

\subtitle{What is Phonetics?}

\author[Brinkerhoff] % (optional, use only with lots of authors)
{Mykel Loren Brinkerhoff}
% - Give the names in the same order as the appear in the paper.
% - Use the \inst{?} command only if the authors have different
%   affiliation.

\institute[UW] % (optional, but mostly needed)
{University of Washington}
% - Use the \inst command only if there are several affiliations.

\date[2025-09-25] % (optional, should be abbreviation of conference name)
{September 25, 2025}




% If you have a file called "university-logo-filename.xxx", where xxx
% is a graphic format that can be processed by latex or pdflatex,
% resp., then you can add a logo as follows:

% \pgfdeclareimage[height=0.5cm]{university-logo}{university-logo-filename}
% \logo{\pgfuseimage{university-logo}}



% Delete this, if you do not want the table of contents to pop up at
% the beginning of each subsection:
% \AtBeginSubsection[]
% {
%   \begin{frame}<beamer>{Outline}
%     \tableofcontents[currentsection,currentsubsection]
%   \end{frame}
% }


% If you wish to uncover everything in a step-wise fashion, uncomment
% the following command: 

%\beamerdefaultoverlayspecification{<+->}


\begin{document}

\begin{frame}
  \titlepage
\end{frame}

%-----------------------------------------------------------
\section{What is Phonetics?}
%-----------------------------------------------------------
\begin{frame}{What is Phonetics?}
    \begin{itemize}
        \item The study of speech sounds.
        \item There are three main areas research in phonetics:
        \begin{enumerate}
            \item Speech \textit{articulation}: how speech sounds are produced using articulators
            \item Speech \textit{acoustics}: what sounds are like in the air
            \item Speech \textit{perception}: how our auditory and cognitive systems interpret sounds
        \end{enumerate}
    \end{itemize}
\end{frame}

\begin{frame}{What is Phonology?}
    \begin{itemize}
        \item \textbf{Phonology} is the other subfield of linguistics that studies speech sounds.
        \item Phonology is concerned with how the sounds we produced are organized in linguistic systems (i.e., Language and languages) and sound structures (e.g., syllables, words, and phrases).
        \item We will cover some phonology in this class, but the main focus is phonetics.
    \end{itemize}
\end{frame}

%-----------------------------------------------------------
\section{What we will cover in this course}
%-----------------------------------------------------------
\begin{frame}{What we will cover in this course}
    \begin{itemize}
        \item Understand the basic concepts and terminology of phonetics.
        \item Learn about the articulatory, acoustic, and perceptual aspects of speech sounds.
        \item Gain familiarity with the methods and tools used in phonetic research.
        \item Explore the relationship between phonetics and other subfields of linguistics.
    \end{itemize}
\end{frame}

%-----------------------------------------------------------
\section{Transcribing speech sounds}
%-----------------------------------------------------------

\begin{frame}{Transcription}
    \begin{itemize}
        \item \textbf{Phonetic transcription} is a way of visually representing speech sounds using symbols.
        \item The most common system for phonetic transcription is the \textbf{International Phonetic Alphabet (IPA)}.
        \item The IPA provides a unique symbol for each distinct sound (or phoneme) in human language.
        \item We will learn how to use the IPA to transcribe speech sounds accurately.
    \end{itemize}
\end{frame}

\begin{frame}{Why do we need a system for phonetic transcription?}
    \begin{center}
        \huge Let's talk about the elephant in the room.
    \end{center}
\end{frame}

\begin{frame}{Why do we need a system for phonetic transcription?}
    \begin{center}
        \huge English orthography is a mess. 
    \end{center}
\end{frame}

\begin{frame}{English orthography}
    \begin{itemize}
        \item English spelling is not a reliable guide to pronunciation; it is often inconsistent and ambiguous.
        \item Is it the worst spelling system? 
        \item Most languages don't require that you do spelling tests in school.
    \end{itemize}
\end{frame}

\begin{frame}{Same spelling, different sounds}
    \begin{itemize}
        \item Examples:
        \begin{itemize}
            \item l\underline{\textbf{ea}}d (to guide) vs l\underline{\textbf{ea}}d (a type of metal)
            \item sh\underline{\textbf{oe}}s vs g\underline{\textbf{oe}}s
            \item t\underline{\textbf{omb}} vs c\underline{\textbf{omb}} vs b\underline{\textbf{omb}}
        \end{itemize}
        \item The same letter or combination of letters can represent different sounds in different words.
        \item A good transcription system would allow us to represent each of these sounds differently.
    \end{itemize}
\end{frame}

\begin{frame}{Different spelling, same sound}
    \begin{itemize}
        \item Examples:
        \begin{itemize}
            \item h\underline{\textbf{ea}}vy, fr\underline{\textbf{ie}}nd, \underline{\textbf{a}}ny, b\underline{\textbf{u}}ry, l\underline{\textbf{eo}}pard (all have the vowel sound ``eh'')
            \item s\underline{\textbf{u}}n, l\underline{\textbf{o}}ve, b\underline{\textbf{u}}tton, c\underline{\textbf{u}}p, d\underline{\textbf{oe}}s, s\underline{\textbf{o}}me, t\underline{\textbf{ou}}ch (all have the vowel sound ``uh'')
        \end{itemize}
        \item The same sound can be represented by different letters or combinations of letters in different words.
        \item A good transcription system would allow us to represent each of these sounds the same way.
    \end{itemize}
\end{frame}

\begin{frame}{``Silent'' letters}
    \begin{itemize}
        \item Examples:
        \begin{itemize}
            \item \underline{\textbf{k}}now, \underline{\textbf{g}}nome
            \item \underline{\textbf{p}}sychology, t\underline{\textbf{h}}yme
            \item \underline{\textbf{w}}rap, \underline{\textbf{h}}our
        \end{itemize}
        \item Some letters in words are not pronounced at all.
        \item A good transcription system would allow us to ignore these letters and only represent the sounds that are actually pronounced.
    \end{itemize}
\end{frame}

\begin{frame}{Dialectal differences}
    \begin{itemize}
        \item Examples:
        \begin{itemize}
            \item The vowel sound in ``cot'' and ``caught'' is pronounced the same in some dialects (e.g., Western American English) but differently in others (e.g., Southern American English).
            \item The final sound in ``car'' is pronounced with a rhotic ``r'' in some dialects (e.g., General American English) but not in others (e.g., British English).
        \end{itemize}
        % On the first slide, make it appear:
        \only<1>{
        % (empty, so nothing appears here)
        }

        % On the second slide, make it disappear (do nothing):
        \only<2>{
        \item The same word can be pronounced differently by speakers of different dialects or even by the same speaker in different contexts. 
        \begin{itemize}
            \item Urban Utahn English pronunciation of the ``t'' in ``mountain'' can vary between a clear ``t'' sound in formal speech and a glottal stop (i.e., the sound you make when you say ``uh-oh'') in casual speech.
        \end{itemize}
        }
        \only<3>{
        \item The same word can be pronounced differently by speakers of different dialects or even by the same speaker in different contexts. 
        \item A good transcription system would allow us to represent these differences accurately.
        }
    \end{itemize}
\end{frame}

\begin{frame}{The Big Idea}
    \begin{center}
        \huge We need a system that allows us to represent each distinct sound with a unique symbol, regardless of how it is spelled or pronounced.
    \end{center}
\end{frame}

\begin{frame}{The Big Idea}
    \begin{center}
        \huge 1 sound = 1 symbol \\
        1 symbol = 1 sound
    \end{center}
\end{frame}

\begin{frame}{Orthography and Transcription}
    \begin{block}{Question}
        Is this a problem for just English; what about other languages (provide examples)?
    \end{block}
    \begin{itemize}
        \item Find two other people in the room and discuss this question for a few minutes.
        \item Appoint someone in your group and be prepared to share your thoughts and/or examples with the class.
    \end{itemize}
\end{frame}

%-----------------------------------------------------------
\section*{The IPA}
%-----------------------------------------------------------
\begin{frame}{The International Phonetic Alphabet (IPA)}
    \begin{itemize}
        \item The IPA is a standardized system of phonetic notation that was initially created by the International Phonetic Association in the late 19th century.\footnote{The most recent version (2018) of the IPA chart can be found on our Canvas page under Resources.}
        \item The IPA provides a unique symbol for each distinct sound in human language.
        \item The IPA is used by linguists, speech-language pathologists, singers, actors, and others who need to accurately represent speech sounds.
        \item BUT: The IPA is not perfect; it has limitations and challenges.
    \end{itemize}
\end{frame}

%-----------------------------------------------------------
\section*{Takeaways}
%-----------------------------------------------------------
\begin{frame}{Your tasks for next time}
    \begin{itemize}
        \item Complete the exit ticket for today on Canvas by 12:30pm.
        \item Read Reetz \& Jongman (2009) Chapter 2.
        \begin{itemize}
            \item There will be a discussion post on Canvas for this reading (This counts as class participation)
        \end{itemize}
        \item Complete Quiz 0 by Friday at 11:59pm.
        \item Complete Homework 1 by Tuesday at 8:30am.
        \item Be ready to dive into Articulation and the IPA next time!
    \end{itemize}
\end{frame}

\begin{frame}
    \begin{center}
        [ hæv ə fʌn wikɛn ɛn wɪl siː ɑl jal ɪn klæs ɑn tjuzdeɪ ]
    \end{center}
\end{frame}

% \subsection<presentation>*{References}
% %-----------------------------------------------------------
% \begin{frame}[t,allowframebreaks]
%   \frametitle<presentation>{References}
%     \printbibliography
% \end{frame}

\end{document}