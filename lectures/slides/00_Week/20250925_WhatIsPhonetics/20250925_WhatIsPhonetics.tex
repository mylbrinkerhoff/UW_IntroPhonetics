% !TEX TS-program = lualatex
% !TEX encoding = UTF-8 Unicode

\documentclass{beamer}

\mode<presentation>
{
  \usetheme{Madrid} % or try Darmstadt, Madrid, Warsaw, ...
  % or ...
  \setbeamertemplate{bibliography item}{}
  \setbeamercovered{transparent}
  % or whatever (possibly just delete it)
}

\usepackage{fontspec}
\usepackage[english]{babel}
% or whatever
\usepackage{csquotes}
\usepackage[backend=biber,
        style=unified,
        maxcitenames=3,
        maxbibnames=99,
        natbib,
        url=false]{biblatex}
% \addbibresource{Dissertation.bib}
\setmainfont{Libertinus Sans} % Main font
% \setsansfont{libertinus sansserif} % Sans serif font
% \setmonofont{CMU Typewriter Text}
\renewcommand{\ttdefault}{cmtt}

% \usepackage[colorlinks,allcolors={black},urlcolor={blue}]{hyperref} %allows for hyperlinks and pdf bookmarks 
\usepackage{graphicx}	%Inserting graphics, pictures, images 		
\usepackage{multicol} %Multicolumn text
\usepackage{multirow} %Useful for combining cells in tablesbrew 
% \usepackage{booktabs} %Enhanced tables
% \usepackage{underscore} %Allows for underscores in text mode
% \usepackage[colorlinks,allcolors={black},urlcolor={blue}]{hyperref} %allows for hyperlinks and pdf bookmarks
\usepackage{url} %allows for urls
\def\UrlBreaks{\do\/\do-} %allows for urls to be broken up
% \usepackage[normalem]{ulem} %strike out text. Handy for syntax
% \usepackage{tcolorbox}
% \usepackage{datetime2}
\usepackage{caption}
\usepackage{subcaption}
\usepackage{langsci-gb4e} % Language Science Press' modification of gb4e
\usepackage{tikz} % Drawing Hasse diagrams
\usetikzlibrary{decorations.pathreplacing}
\usepackage{leipzig} %	Offers support for Leipzig Glossing Rules


\title[LING 450/550] % (optional, use only with long paper titles)
{Introduction to Linguistic Phonetics}

\subtitle{What is Phonetics?}

\author[Brinkerhoff] % (optional, use only with lots of authors)
{Mykel Loren Brinkerhoff}
% - Give the names in the same order as the appear in the paper.
% - Use the \inst{?} command only if the authors have different
%   affiliation.

\institute[UW] % (optional, but mostly needed)
{University of Washington}
% - Use the \inst command only if there are several affiliations.

\date[2025-09-25] % (optional, should be abbreviation of conference name)
{September 25, 2025}




% If you have a file called "university-logo-filename.xxx", where xxx
% is a graphic format that can be processed by latex or pdflatex,
% resp., then you can add a logo as follows:

% \pgfdeclareimage[height=0.5cm]{university-logo}{university-logo-filename}
% \logo{\pgfuseimage{university-logo}}



% Delete this, if you do not want the table of contents to pop up at
% the beginning of each subsection:
% \AtBeginSubsection[]
% {
%   \begin{frame}<beamer>{Outline}
%     \tableofcontents[currentsection,currentsubsection]
%   \end{frame}
% }


% If you wish to uncover everything in a step-wise fashion, uncomment
% the following command: 

%\beamerdefaultoverlayspecification{<+->}


\begin{document}

\begin{frame}
  \titlepage
\end{frame}

%-----------------------------------------------------------
\section{What is Phonetics?}
%-----------------------------------------------------------
\begin{frame}{What is Phonetics?}
    \begin{itemize}
        \item The study of speech sounds.
        \item There are three main areas research in phonetics:
        \begin{enumerate}
            \item Speech \textit{articulation}: how speech sounds are produced using articulators
            \item Speech \textit{acoustics}: what sounds are like in the air
            \item Speech \textit{perception}: how our auditory and cognitive systems interpret sounds
        \end{enumerate}
    \end{itemize}
\end{frame}

\begin{frame}{What is Phonology?}
    \begin{itemize}
        \item \textbf{Phonology} is the other subfield of linguistics that studies speech sounds.
        \item Phonology is concerned with how the sounds we produced are organized in linguistic systems (i.e., Language and languages) and sound structures (e.g., syllables, words, and phrases).
        \item We will cover some phonology in this class, but the main focus is phonetics.
    \end{itemize}
\end{frame}

%-----------------------------------------------------------
\section{What we will cover in this course}
%-----------------------------------------------------------
\begin{frame}{What we will cover in this course}
    \begin{itemize}
        \item Understand the basic concepts and terminology of phonetics.
        \item Learn about the articulatory, acoustic, and perceptual aspects of speech sounds.
        \item Gain familiarity with the methods and tools used in phonetic research.
        \item Explore the relationship between phonetics and other subfields of linguistics.
    \end{itemize}
\end{frame}

%-----------------------------------------------------------
\section{Transcribing speech sounds}
%-----------------------------------------------------------

\begin{frame}{Transcription}
    \begin{itemize}
        \item \textbf{Phonetic transcription} is a way of visually representing speech sounds using symbols.
        \item The most common system for phonetic transcription is the \textbf{International Phonetic Alphabet (IPA)}.
        \item The IPA provides a unique symbol for each distinct sound (or phoneme) in human language.
        \item We will learn how to use the IPA to transcribe speech sounds accurately.
    \end{itemize}
\end{frame}

\begin{frame}{Why do we need a system for phonetic transcription?}
    \begin{center}
        \huge Let's talk about the elephant in the room.
    \end{center}
\end{frame}

\begin{frame}{Why do we need a system for phonetic transcription?}
    \begin{center}
        \huge English orthography is a mess. 
    \end{center}
\end{frame}

\begin{frame}{English orthography}
    \begin{itemize}
        \item English spelling is not a reliable guide to pronunciation; it is often inconsistent and ambiguous.
        \item Is it the worst spelling system? 
        \item Most languages don't require that you do spelling tests in school.
    \end{itemize}
\end{frame}


% \begin{frame}{Why do we need a system for phonetic transcription?}
%     \begin{itemize}
%         \item For example, the letter "c" can represent different sounds in different words:
%         \begin{itemize}
%             \item "cat" /kæt/ (the "c" represents the sound /k/)
%             \item "city" /ˈsɪti/ (the "c" represents the sound /s/)
%             \item "chocolate" /ˈtʃɒklət/ (the "ch" represents the sound /tʃ/)
%         \end{itemize}
%     \end{itemize}
% \end{frame}

% \subsection<presentation>*{References}
% %-----------------------------------------------------------
% \begin{frame}[t,allowframebreaks]
%   \frametitle<presentation>{References}
%     \printbibliography
% \end{frame}

\end{document}